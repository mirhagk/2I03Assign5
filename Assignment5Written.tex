\newcommand{\code}{\texttt}
\newcommand{\PP}{\emph{Pair Programming} }
\newcommand{\CR}{\emph{Code Review} }

\documentclass{article}

\usepackage{cite}

\title{Code Review vs. Pair Programming}
\author{Nathan Jervis \and Victor Reginato \and Matt Pagnan}

\begin{document}

\maketitle

\section{Introduction}

\subsection{Background}
A huge problem that many software companies are running into is that as their programmers leave the company the code that they had written becomes unmanageable since no one know what the code does exactly. This is because at the time code was written to be able accomplish the task that needed to be done and no one bothered to make their code so that others we able to read and understand what their code did.

The solution to combat unreadable code was to have more than one person work on the code. The thought process behind this was that if more than one person worked to create the code then the code would be much more readable than if only one person worked on the code. This is because in order to make the code more than one person had to understand what was going on while making the code. There are multiple method of doing this each has its own advantages and disadvantages. In this report we are going to compare \PP and \CR.

In this document we shall be comparing \PP and \CR. We will briefly give an overview of what \PP is and what \CR is then we will compare the two in the following categories.
\begin{itemize}
\item Code Quality
\item Cost and Productivity
\item Difficulty in managing
\end{itemize}

In each of these categories we will talk about what is similar and what is different between \PP and \CR 

\section{Code Review and Pair Programming}

\subsection{\PP}
\PP is where two programmers work together on the same project, at the same workstation, on the same computer. One programmer writes the code while the other watches them. After a while both the programmers will swap roles. The programmer who is watching is able to see what the other one is doing and come up with ideas and improvements for when it is their turn. This leaves the other programer to code away with out much worry of errors as they know their partner is keeping an eye on things and will fix them when they get their turn to code.

\subsection{\CR}
\CR is when one programmer works on a project then later sends their project to another programmer to review thier 


\subsection{Code Quality}

Code

\subsection{Cost and Productivity}

Software developers are not free, nor are they cheap, so cost is a big factor in choosing the software design process. An inexperienced manager who doesn't understand programming might see both processes as additional expense.Upon suggesting \PP to management, many software developers are faced with "But then it'll cost twice as much", or equivalently "But then productivity will halve". \CR is also seen as wasting time as well, but it's usually easier to justify, and isn't usually seen as doubling time amount of time required.

Despite what some might think, neither one doubles cost, or halves productivity. In the long run both of them increase productivity because they increase code quality, and increase the number of developers who understand a section of code. Having multiple developers understand a section of code means that if one developer goes on vacation or leaves the workplace, the code can still be maintained and fixed.

Catching errors as soon in the development process as you can reduces the cost to fix the bug. \cite{avgCostOfDefect}\footnote{Table 3.1 Code Complete} outlines the costs for fixing bugs caught during various parts of the software development process. A bug introduced during construction (ie programming error) costs 10 times as much if caught during system testing than if caught during construction. The same bug can be 10-25 times as expensive if caught after release. This suggests that if the majority of a project's time is spent fixing bugs, introducing code review or pair programming to reduce the number of bugs could increase productivity by 10.

Which of the 2 efforts costs more\footnote{in terms of after bug-cost, and lost developer time trying to understand poorly designed code}? It's dependent on what systems are already in place for your software team. 

\PP requires two programmers working together on the same code, with only one writing at a time. This naturally works best when the team is located in the same area, and the workplace is comfortable enough to have both programmers working at a time. Small cubicles don't work well, as both programmers will get too crowded, unless pair programming is only down in small sprees, in which case the full benefits aren't realized. \PP could be implemented using something similar to Skype, but then bad network connections and poor sound quality have a serious effect on productivity. As such the cost for \PP could be very high, as it could mean changing the workspace of each worker significantly.

\CR often requires much less to incorporate into design practices. The neccessary infratrustructe is already in place with source control systems\footnote{If source control isn't currently used, stop reading right now and implement source control before worrying about anything else. It's by far the most important thing to do if you want quality code}


\subsection{Difficulty in Managing}

\section{Conclusion}


\bibliography{WA5BIB}
\bibliographystyle{plain}

\end{document}