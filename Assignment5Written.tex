\newcommand{\code}{\texttt}
\newcommand{\PP}{\emph{Pair Programming} }
\newcommand{\PR}{\textbf{\huge{HEY VICTOR DON'T BE DUMB PR DOESN'T EVEN EXIST}}}
\newcommand{\CR}{\emph{Code Review} }


\documentclass{article}

\usepackage{cite}

\title{Code Review vs. Pair Programming}
\author{Nathan Jervis \and Victor Reginato \and Matt Pagnan}

\begin{document}

\maketitle

\tableofcontents

\section{Introduction}

\subsection{Background}
A huge problem that many software companies are running into is that as their programmers leave the company the code that they had written becomes unmanageable since no one know what the code does exactly. This is because at the time code was written to be able accomplish the task that needed to be done and no one bothered to make their code so that others we able to read and understand what their code did.

The solution to combat unreadable code was to have more than one person work on the code. The thought process behind this was that if more than one person worked to create the code then the code would be much more readable than if only one person worked on the code. This is because in order to make the code more than one person had to understand what was going on while making the code. There are multiple method of doing this each has its own advantages and disadvantages. In this report we are going to compare \PP and \CR.

In this document we shall be comparing \PP and \CR. We will briefly give an overview of what \PP is and what \CR is then we will compare the two in the following categories.
\begin{itemize}
\item Code Quality
\item Cost and Productivity
\item Difficulty in managing
\end{itemize}

In each of these categories we will talk about what is similar and what is different between \PP and \CR

\subsection{The Need for Review}
\PP and \CR are both programming styles that focus on the collaboration of at least two programmers, however both have their individual strengths and weaknesses. Firms may want to know how to best allocate the time of their programmers. This document will provide an analysis and breakdown of both to help determine the programming style that best suits your programming needs.

\section{Code Review and Pair Programming}

\subsection{\PP}
\PP is where two programmers work together on the same project, at the same workstation, on the same computer. One programmer writes the code while the other watches them. After a while both the programmers will swap roles. The programmer who is watching is able to see what the other one is doing and come up with ideas and improvements for when it is their turn. This leaves the other programmer to code away with out much worry of errors as they know their partner is keeping an eye on things and will fix them when they get their turn to code.

\subsection{\CR}
\CR is when programmer one works on a project and then later sends their project to programmer two who will look over their code. Programmer one and Programmer two will have their own separate workstations. While programmer one is working on their project programmer two can be doing something else, such as working on their own project. When programmer two is reviewing programmer one's project programmer one is free to do something else, usually if programmer two was working on their own project programmer one would then review programmer two's project. 

\subsection{Code Quality}

Both approaches increase the quality of code by a significant amount. Code Complete\cite[Table 20-2]{McConnell:2004:CCS:1096143} found that various practices discovered more defects than others. \ref{tab:defect} Lists their conclusions.

\begin{table}[hb]
\begin{tabular}{c|c}
Method & Percent of Defects Found\\\hline
Unit Testing & 30\% \\
Function Testing & 30\% \\
Integration Testing & 35\% \\
Informal Code Review & 25\% \\
Formal Code Review & 60\%\\
\end{tabular}
\label{tab:defect}
\caption{Percentage of defects discovered by various methods}
\end{table}

From this table we can see that coder reviews are able to find about 60\% of defects, compared to 30\% from testing functionality\footnote{Note that it also discovered that informal code reviews only discover about 25\%, so a formal process needs to be in place}. Not only that but \CR and \PP also increase the readability of the code, promote clearer variable and function names, and more comments.

The big difference between the two lies in whether the code is able to speak for itself or not. In order for code to pass a \CR it must be clear from the code and comments alone what exactly it does. The original programmer isn't there to shed additional light, so if code isn't clear, it is rejected and the original programmer has to make it clear before it passes review. With \PP the original programmer is sitting there with the reviewer, and the information known to the original programmer is also known to the reviewer. This means that the reviewer may take for granted some information that might not be known to another developer on the team looking at the code later. It is for this reason that \CR could produce clearer code, and with \PP the reviewer needs to be on the lookout for this.

\subsection{Cost and Productivity}

Software developers are not free, nor are they cheap, so cost is a big factor in choosing the software design process. An inexperienced manager who doesn't understand programming might see both processes as additional expense.Upon suggesting \PP to management, many software developers are faced with "But then it'll cost twice as much", or equivalently "But then productivity will halve". \CR is also seen as wasting time as well, but it's usually easier to justify, and isn't usually seen as doubling time amount of time required.

Despite what some might think, neither one doubles cost, or halves productivity. In the long run both of them increase productivity because they increase code quality, and increase the number of developers who understand a section of code. Having multiple developers understand a section of code means that if one developer goes on vacation or leaves the workplace, the code can still be maintained and fixed.

Catching errors as soon in the development process as you can reduces the cost to fix the bug. Code Complete\cite[Table 3.1]{McConnell:2004:CCS:1096143} outlines the costs for fixing bugs caught during various parts of the software development process. A bug introduced during construction (ie programming error) costs 10 times as much if caught during system testing than if caught during construction. The same bug can be 10-25 times as expensive if caught after release. This suggests that if the majority of a project's time is spent fixing bugs, introducing code review or pair programming to reduce the number of bugs could increase productivity by 10.

\PP also allows programmers to catch each other's mistakes, and remember method names or uses without looking at doucmentation. During a study comparing individual programmers to pairs of programmers, it was discovered that 60\% more time was spent by the teams (when combining their time). After the first task completed, and the programmers were adjusted, that time reduced to as little as 15% more time required. This means that the costs for programmers in the short term will increase between 15-60\%, and as established the high code quality will decrease the cost in the long run.

Which of the 2 efforts costs more\footnote{in terms of after bug-cost, and lost developer time trying to understand poorly designed code}? It's dependent on what systems are already in place for your software team. 

\PP requires two programmers working together on the same code, with only one writing at a time. This naturally works best when the team is located in the same area, and the workplace is comfortable enough to have both programmers working at a time. Small cubicles don't work well, as both programmers will get too crowded, unless pair programming is only down in small sprees, in which case the full benefits aren't realized. \PP could be implemented using something similar to Skype, but then bad network connections and poor sound quality have a serious effect on productivity. As such the cost for \PP could be very high, as it could mean changing the workspace of each worker significantly.

\CR often requires much less to incorporate into design practices. The necessary infrastructure is already in place with source control systems\footnote{If source control isn't currently used, stop reading right now and implement source control before worrying about anything else. It's by far the most important thing to do if you want quality code}, so the cost to switch is just in training your developers to perform them. In this way \CR could be cheaper than \PP, but if \PP is found to be more productive for your team, it may be better. The only problem is \PP potentially costs a significant amount in order to try it, while \CR can be tried out for essentially no cost. If the management is worried about the costs associated with \PP, it may be easier to try out \CR as the lower cost will mean the benefits will come sooner. In an ideal situation both should be performed to determine the costs, as it will vary based on the situation, the team and the project.

\subsection{Difficulty in Managing}

In the theoretical world management could choose any software practice it wanted, and all the developers would automatically be capable and willing to follow. In practice however developers may have difficulties understanding what is expected, or might even be lazy about doing their parts.

\section{Conclusion}


When possible, both practices should be implemented in order to have the highest quality code. The benefits of \PP and \CR are noticed the most in the decreased debugging time and overall code quality. The likely hood of launch errors is significantly decreased and the coding process is generally more efficient. However, it is not always possible to have both styles of programming be implemented due to the amount of resources (programmers) it occupies. In the is case where one or the other must be picked due to resource constraints, it is best to weigh the advantages and disadvantages of both.

If the project in question is preforming complicated and convoluted data manipulation which is not necessarily done locally, then \PP will show its strengths. These strengths include things like pair-pressure\cite{William} - the pressure created by having another programmer that wants to stick to standard and maintainable practices. The presence of another programmers ensures two very important things: that the code produced will be more readable; that there will be another programmer who understands the code and can maintain it. Having maintainable code ensures that future employees or different coders within the company can understand and implement changes to the existing system. 

\CR shows its strengths when the code that is being designed is more simplistic, and shouldn't be a problem that one programmer should struggle solving. The major benefit of code review is that if there are any errors, the coder reviewing the code should pick up on them fairly quickly. This also means that you have two programmers that understand the code being developed.This ensures that there is another coder that can work on the project. It consumes less of the programmers time to review code, and with that saved time, they can work on other projects. This is a very efficient way to improve code quality and and good use of company resources. 



%CITATIONS, TOO LAZY TO DO RIGHT NOW 
%http://www.cs.utah.edu/~lwilliam/Papers/ieeeSoftware.PDF  , Strengthening the Case for Pair-Programming 
%Table 3.1 Code Complete
%http://www.codinghorror.com/blog/2006/01/code-reviews-just-do-it.html (Actual source is Code Complete)


\bibliography{WA5BIB}
\bibliographystyle{plain}

\end{document}